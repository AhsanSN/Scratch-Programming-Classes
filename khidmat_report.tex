\documentclass{article}

\usepackage{geometry}
\usepackage{hyperref}

\title {Khidmat: Project Title}

\author{
  Student1 Full Name\\ ab01234
  \and
  Student2 Full Name\\ xy09876
}
\date{}  

\begin{document}
\maketitle

% Use first person plural (we, us) even if you did the Khidmat individually.

% An introduction of the project, no more than 2 sentences. Provide the highest level of detail only. Other details will come later.
% Typically, "This project is to <short description of porject> for/at <client>."
This project is to build a testing system to be used for the entrance examination at Habib University.

% About the client.
Habib University is a first of its kind liberal arts and science university in Pakistan. It offers 4 major and several minor programs. Founded through the largest philanthropic grant to higher education in the history of Pakistan, it is a registered non-profit organization with \href{http://www.pcp.org.pk}{Pakistan Centre for Philanthropy (PCP)}.

% About the project.
Striving to admit the most deserving students, Habib University uses \href{https://accuplacer.collegeboard.org}{ACCUPLACER} for its entrance test but wants to move to its own test. Faculty members will contribute questions to a pool and the new examination system will choose and present questions at random from the pool to each test taker. This will ensure that each test taker gets a unique test. For our Khidmat, we will build Habib University's new examination system.

% About the plan of work.
We will work full time on Habib University's premises under the supervision of their Admission Manager. The goal is to develop, test, and deploy the system by the end of our Khidmat.

% Copy-paste this section with necessary modifcations for each week.
\newpage % Start the report for each week on a new page.
\section*{Week 1: 9--14 July, 2018}

% A summary, maximum 2 sentences, of this week's activities.
We spent this week meeting several stakeholders in order to understand the shortcomings of ACCUPLACER and the expectations from the new system.

\begin{tabular}{|l|l|l|l|}
  \hline
  Item 	& Activity & Time & ID \\\hline\hline
  1	& Met Admissions team & 3 hrs & st1 \\\hline
  2	& Met faculty & 2 hrs & st2 \\\hline
  3	& Met IT team & 3 hrs & st1, st2 \\\hline
  $\vdots$ & $\vdots$ & $\vdots$ & $\vdots$ \\\hline
\end{tabular}

The total time spent on the Khidmat this week is as follows.

\begin{tabular}{|l|l|}
  \hline
  ID & Total Hours\\\hline\hline
  st1 & 7 hours\\\hline
  st2 & 6 hours\\\hline
\end{tabular}

% Other weeks ...

\newpage
\section*{Conclusion}

% Remind the reader about the project. Summarise your activities over the course of the project.
Our project was to build a new testing system for Habib University to replace ACCUPLACER for its entrance examination. We started by meeting all the stakeholders to understand their expectations from the new system. We then identified the necessary tools to build the required system and trained ourselves on them. Development and testing were carried out in collaboration with the IT team so that any shortcomings were identified and catered to as we went along. The system was then deployed and officers from the Admissions Team were trained to use it.

\newpage
% Show your external supervisor your report, especially the weekly upates; have them sign a printed copy of this page; scan the signed page; and include the scanned page in this document as an image.

I hereby certify that I supervised this Khidmat and that I have read and agree with the weekly updates included in the Khidmat report.\\[50pt]

\noindent\begin{tabular}{@{}p{.6\textwidth}@{\hspace{.1\textwidth}}p{.3\textwidth}}
  \hrulefill &   \hrulefill \\
  Name and signature & Date and place
\end{tabular}

\end{document}
